\documentclass[runningheads,orivec]{llncs} 
\usepackage{amsmath} 
\usepackage[pipeTables,citations,tableCaptions]{markdown} 
\usepackage[T1]{fontenc} 
\usepackage{graphicx} 
\setkeys{Gin}{width=0.75\linewidth}
\setlength{\intextsep}{0.3em plus 0em minus 0.3em}
\usepackage{url}
\usepackage[hidelinks]{hyperref}

\begin{document} 
\title{Machine learning for sustainable UK groundwater} 
\author{James W.G. Carlyle\inst{1}} 
\authorrunning{J.W.G. Carlyle} 
\institute{University of Southampton} 
\maketitle

\begin{abstract} 
    A high-resolution UK hydrogeological map (the British Groundwater Model) now exists, but uses a physics computational model and is frequently inaccurate. There is an opportunity to examine whether machine learning (ML) techniques can improve groundwater height and flow prediction at a national scale. The societal benefits are sustainable water extraction for housing and datacentres, reduced flooding and drought, and improved ecology of rivers and wetlands. 
    \keywords{Groundwater \and Machine Learning \and Flood \and Drought} 
\end{abstract}

\begin{markdown} 
## Introduction and Motivation 

Water is a fundamental resource for terrestrial wildlife and human society. It is essential for all agriculture, and has a huge impact on human health. I worked in Nepal where contaminated groundwater led to a 25\% mortality rate in children below the age of 5 [@NationalStatisticsOffice2025].  Although clean water provision around the world has improved since then, it remains critical for good health. Growing populations and a drive for higher living standards make groundwater extraction unsustainable.

Water is abstracted from rivers, lakes, and the ground, which holds about 30\% of the world's freshwater [@Jomaa2025]. Groundwater has several major advantages - it's a natural reservoir, and is lower in contaminants such as suspended solids and pathogens, resulting in up to 31\% less energy processing to reach drinking quality [@MO20115577]. Groundwater supplies nearly half of the world's drinking water [@Tao2022], and depending on climate, groundwater provides 33-59\% (Europe, North America) of water for agriculture [@UnitedNations2022], [@EEAWAT0072025].

Global groundwater extraction doubled between 1960-2000 [@Wada2010], with satellites such as GRACE-FO showing aquifer depletion exceeding 50 metres in critical regions. In India, 17 km$^3$ has been lost in a single year [@dalin2017groundwater]. Analysis by IPCC suggests that groundwater removal may account for 15\% of global sea-level rise [@GRACEGroundwater], because 80\% of extracted groundwater remains in the oceans [@Wada2016].

Groundwater recharge needs steady rainfall, but in the UK, rainfall is becoming more sporadic and intense, with rain exceeding 20 mm/h expected four times more often by 2070 [@Kendon2023]. Infiltration rates can halve under dry and compacted soils [@reinsch2024temperate], and so lower groundwater increases the probability of future drought. Subsequent rainfall which isn't adsorbed into the surface causes flood damage, and the run-off overwhelms sewage systems, causing 450,000 sewage discharges in the UK in 2024 [@riverstrust2024sewage] and leading to further environmental damage. Despite this, up to 15\% of England experienced freshwater usage restrictions in 2025 [@Independent2025].

Changes in river flow caused by extraction affect ecological indicators, including invertebrate diversity, algae, bacteria growth, and fish biomass. For example, extraction leads to increased algae biomass but reduced invertebrate richness and organic matter decomposition [@bradley2012ecological].

Effective management of groundwater requires predictive capacity. But groundwater systems present inherently complex prediction problems, as they are influenced by numerous interacting factors, from meteorological variables (precipitation, temperature, evaporation) to human activities (extraction and land use). Groundwater is difficult to model physically since aquifers contain unknown geological discontinuities. Identifying hydrogeological properties is complex because it requires detailed subsurface data, which is difficult to obtain comprehensively [@condon2021global].

## National-scale Groundwater Modelling

### A Case Study of the UK

Groundwater is modelled by physics-based software such as MODFLOW, a 40 year-old finite-difference model written in Fortran, and this demands extensive calibration and knowledge of the heterogeneous geology [@hill2007effective]. Conventional approaches often struggle to model the non-linear characteristics of real-world groundwater systems effectively, as predictions are generated by heuristic methods, rather than observed inputs and outputs [@acerotriana2019beyond].

The UK's hydrogeology is defined by significant spatial variation:

|**Region**         |**Dominant Geology**      |**Groundwater Importance**  | 
|-------------------|--------------------------|----------------------------| 
|South East England |Chalk, Triassic Sandstone |Very high (>75\%) [@condon2021global]    | 
|                   |Productive fissured       |                            | 
|East Midlands      |Jurassic Limestone        |Moderate to high            | 
|                   |Fissured/inter-granular   |                            | 
|Wales & Scotland   |Ordovician, Devonian, etc.|Low (ca. 5\%) [@condon2021global]        | 
|                   |Low permeability          |                            | 
|Northern England   |Carboniferous Limestone   |Moderate                    |                     
|                   |Rapid fissured            |                            |

: Geology and groundwater importance in the UK.

The British Geological Survey (BGS) has produced the British Groundwater Model (BGWM) [@hydrojules2022britishgwmodel], covering the whole of Great Britain down to 1km scale to simulate transient groundwater dynamics and surface water/groundwater interaction. Such models also exist for other countries, such as China [@yang2025concn]. The purpose of the BGWM is to answer questions such as:

* How can an integrated approach improve the simulation of major flooding events such as the 2013-14 floods?
* How can a holistic approach be undertaken to assess water resources under drought conditions?

![Hydro-JULES grid mapping for BGWM.](../images/uk-geological-grid.png 'Hydro-JULES grid mapping for BGWM.')

### Gaps / Issues with the BGWM

BGWM is unreliable in several areas [@Bianchi2024BGWM], such as:

* Where observation coverage is sparse, simulated heads have high variance (standard deviation often >10 m), making results less reliable.

* Carboniferous Limestone areas show high standard deviations in simulated heads, again indicating greater calibration uncertainty.

* Steep relief areas with strong topographic gradients consistently have larger head uncertainty, so local groundwater levels there are poorly constrained.

* Deep flow in limestones and hard rocks: The model assumes groundwater flow  down to 500 m everywhere, which is likely invalid for limestone aquifers and crystalline/metamorphic rocks, where active flow is mostly within a few tens of meters of surface fractures.

* Small‑scale heterogeneities (karst features, clay interbeds, tectonic structures) are not resolved; this particularly limits performance in karstic limestones and structurally complex sandstones.

* Baseflow in large catchments and peak events: The model tends to underestimate peak baseflows in larger catchments.

This is evident in the map of error for heads and baseflow at sampling sites across the UK, where a significant number of sits show a >50\% mean error rate.

![Observed simulated map of BGWM](../images/observed-simulated-map-bgwm.png 'BGWM observed vs simulated map of heads and baseflow')

### Improving Accuracy at National Scale

Machine learning (ML) can capture complex patterns and non-linear relationships without requiring explicit parametrization of underlying hydrogeology processes. Recent advances in ML architectures, including convolutional neural networks (CNNs), recurrent neural networks (RNNs), and spatial-temporal graph neural networks (ST-GNNs), have demonstrated superior predictive accuracy compared to both traditional shallow networks and conventional numerical methods [@elsheikh2022groundwater].

Critically, ML techniques show particular promise in sparse data scenarios, handling missing values and capturing the complex spatial-temporal dependencies that characterize interconnected groundwater systems [@shu2025missing]. For example, there is often a lag between rainfall and groundwater change, as different groundwater systems percolate at different rates.

Recent advances in ML methods have not been tested at national scale. An obvious opportunity exists to apply AI/ML techniques to enhance the predictions provided by purely physics-based computational models.

#### Research Question 1: Which ML techniques are appropriate for national-scale groundwater modelling, and how much can they improve on the prediction accuracy of the physical models currently used?

### Review of Prior ML Techniques

This review classifies prior advances by the technology used. 

#### Support Vector Machines (SVM)

Support vector machines emerged as a powerful alternative to early neural networks in the early 2000s. Yoon˙ et al. [@yoon2011comparative] conducted a comparative study demonstrating that SVMs outperformed traditional NNs for predicting groundwater levels in coastal aquifers. The technique excelled at handling non-linear relationships by mapping inputs to higher-dimensional spaces using kernel functions, particularly the radial basis function (RBF) kernel.

#### Random Forest (RF)

Random forest algorithms have become widely adopted for groundwater modelling due to their ensemble approach and ability to handle high-dimensional datasets while mitigating over-fitting. The method constructs multiple random decision trees using bootstrap sampling and aggregates their predictions with voting to improve robustness. The original paper by Breiman˙ [@breiman2001random] was considered transformational, and more recently RFs have been shown to support effective spatial prediction [@rohde2021mlgwdepth].

#### Ensemble Learning: XGBoost, CatBoost, and Gradient Boosting

Modern ensemble boosting methods have demonstrated performance in groundwater applications. Chen˙ and Guestrin˙ developed XGBoost, which uses regularized model formalization to control over-fitting [@chen2016xgboost]. It has shown superior prediction in multiple comparative studies [@xiong2023spatial], and a comprehensive study of six ensemble models (RF, AdaBoost, XGBoost, CatBoost, GBDT, and LightGBM) again identified XGBoost as having superior performance in groundwater potential mapping [@uddin2025ensemble]. 

#### Early Neural Networks

The application of NNs in groundwater modelling gained prominence in the early 1990s [@maier1996use]. Maier˙ and Dandy˙ published a seminal review [@maier2000neural], establishing fundamental guidelines for developing neural network models for water resource variables. This outlined essential development stages including performance criteria selection, data division and preprocessing, input variable determination, network architecture design, and model validation.

Coulibaly˙ et al. calibrated three types of NN model using groundwater level and hydro-meteorological data, establishing early benchmarks for model performance [@coulibaly2001improving]. Lallahem et al. evaluated neural networks for groundwater level estimation in fractured media [@lallahem2005use]. Their influential research demonstrated that NNs with appropriate architecture could effectively model groundwater systems with R² values approaching 0.95.

Nayak˙ et al. conducted groundwater level forecasting in shallow aquifers using NN approaches, comparing multiple back-propagation algorithms. The Levenberg˙-Marquardt˙ algorithm emerged as the best performer, showing superior results with minimum deviation values mostly within ±0.5 meters and acceptable prediction accuracy exceeding 94.52\% [@nayak2006groundwater].

#### Long Short-Term Memory (LSTM) Networks

The introduction of LSTM networks to hydrology marked a paradigm shift. Kratzert˙ et al. [@kratzert2018rainfall] published a landmark paper applying LSTM networks to rainfall run-off modelling across 241 catchments. This study demonstrated that LSTMs could learn long-term dependencies essential for modelling storage effects in catchments, outperforming the well-known Sacramento Soil Moisture Accounting Model (SAC-SMA).

Zhang˙ et al. [@zhang2018developing] developed an LSTM-based model for predicting water table depth in agricultural areas, achieving remarkable accuracy. The LSTM model demonstrated R² scores ranging from 0.789 to 0.952, substantially outperforming traditional feed-forward neural networks. This work showed that dropout methods effectively prevented over-fitting and that LSTM architecture possessed strong learning ability for time-series groundwater data.

Solgi˙ et al. [@solgi2021lstmnn] demonstrated that LSTM networks could predict groundwater levels with exceptional accuracy: R² of at least 99.89\% for one-cycle predictions and 90\% for 26-cycle ahead predictions over 17 years of data. The study confirmed LSTM's superiority over feed-forward neural networks in both point prediction and prediction interval tasks.

#### Convolutional Neural Networks (CNN)

Seo˙ and Lee˙ [@seo2021predicting] advanced groundwater modelling by developing CNN-LSTM hybrid architectures, combining CNN's spatial feature extraction with LSTM's temporal dependency handling. The C-LSTM model produced more accurate groundwater simulations than stand-alone CNN or LSTM and showed that the use of parameters recorded by GRACE-FO (gravity detection) satellites as training data for deep learning models could play an important role in the model's performance.

### Technique Summary

A recent systematic review of neural network applications for groundwater level prediction [@afful2025review] provides a comprehensive summary of the state-of-the-art. One conclusion is that a fundamental limitation is insufficient training data, as the domain lacks the extensive, labelled datasets from the computer vision or natural language processing domains. Long-term historical data, ideally for 10 years, is essential for reliable groundwater level (GWL) forecasting. This scarcity is particularly acute in developing countries where monitoring infrastructure remains limited or unreliable.

### Methodology
This wide field of research will be tackled with a measured and incremental approach:
1. Assess the architecture and interface for the current MODFLOW implementation of BGWM.
2. Work with a limited time and spatial subset of the underlying datasets (groundwater levels and geology). Pick two regions; one where the model performs well, and one where it underperforms.
3. Determine 3 high-performing ML approaches, following an in-depth review of published results.
4. Apply the 3 chosen ML approaches to the BGWM subsets and evaluate the models and parameters with highest predictive accuracy.
5. Evaluate the selected approach against baseline MODFLOW prediction.
6. Apply the selected approach at a national scale.
7. Re-evaluate, noting and addressing scaling issues, particularly in relation to the following sub-questions:

#### Resolution
What is the best resolution available using appropriately pre-trained ML inference for national-scale groundwater prediction using desktop hardware? A correlation is expected between granularity (resolution) and the computational resource needed when using an established ML model. In other words, high-resolution models need more training and inference computing power. This question can be reframed as: How is it possible to make simulation runs available to scientists with lower-powered computers, such as desktop or field systems?

#### Adaptability to different countries
Can transfer learning of model weights be used to operate in a region with little data? Challenges exist in countries with rapidly expanding populations and rising living standards such as northern India, where levels are dropping 30cm/year and a 6m height anomaly is seen in images produced by NASA. In developing countries there is generally less historical / geological data, so how can ML models be adapted so that they function well in data-constrained regions?

## Hybrid Neural Network Models

Groundwater flow and height depend on physical laws (e.g. fluid flow in porous bodies), space (e.g. location of extraction points, rivers and rainfall catchment boundaries), and time (e.g. the elapsed time between rainfall soaking into the ground and reemerging in a river). 

There is a research gap in adding physics-based loss functions to spatial-temporal graph neural networks to create a PI-ST-GNN implementation for groundwater.
 
#### Research Question 2: Can physical models be added to the loss function of the latest spatial-temporal graph neural networks?

### Review of Prior ML Techniques

To date, research has examined physics-informed and graph neural network models separately, but not together, and not at a national scale.

#### Physics-Informed Networks
Goswami˙ et al. [@taccari2023emulator] introduced DeepONet (deep operator networks) to groundwater modelling, demonstrating that AI-based approaches could successfully address the complexities of simulating subsurface flows. DeepONet uses two deep neural networks (branch and trunk) to learn mappings between infinite-dimensional function spaces.

#### Spatial Temporal Graph Neural Networks
Graph neural networks reflect a physical topography through a graph representation of edges and nodes. 
Taccari˙ et al. [@taccari2024stgnn] pioneered the application of spatial-temporal graph neural networks (ST-GNNs) to groundwater level prediction. This approach addressed the limitation of previous LSTM methods that predominantly focused on temporal dynamics while overlooking spatial relationships. The ST-GNN model showed significant improvements, particularly for handling missing data and long-term forecasting with minimal bias. The graph-based framework integrated physical borehole interconnectivity and temporal aspects, capturing complex interactions within groundwater systems over distance and time.

### Methodology
The proposed research approach uses existing techniques as the control to evaluate against:
1. Analyse a spatial-temporal (ST) network implementation, looking at the structure of the loss function.
2. Look at previous PINNs, and document how physical hydrology laws and geology (e.g. rock porosity) are implemented in loss calculation.
3. Augment the ST network loss-function with physics-derived loss functions.
4. Work with a limited time and spatial subset of the underlying datasets (groundwater levels and geology). 
5. Evaluate the prediction accuracy of the PI-ST network against both the PI network and ST network for the chosen region.

## Conclusion

There is a tremendous opportunity to use ML to better predict groundwater levels across the UK provided by new national-scale geological and hydrological mapping. This could improve the prediction of both drought (seen at scale about once every 10 years) and flooding (about 50 times since 2017). Furthermore, a generalised model could be applied to other countries and regions, provided that a similar form of geological mapping and rainfall data input is available. 

This work has benefit for all societal groups. Although those from lower socio-economic groups do not necessarily experience greater flood risk, they have less ability to prepare for or recover from flood damage, and the financial consequences of flooding are much greater. The research is consumptive in computing power and energy use, and care will be taken to make analysis efficient. The research does not handle personalised data, and no other ethical concerns have been identified, although a full assessment will be undertaken.

\end{markdown} 

\bibliographystyle{splncs04}
\bibliography{main}

\end{document}